\section{SVM}
\subsection{Cost Function}
\begin{equation}\begin{aligned}
	J(\theta) = C\sum_{i=1}^m y^{(i)}cost_1(\theta^Tx^{(i)}) + (1-y^{(i)})cost_0(\theta^Tx^{(i)}) + \frac{1}{2}\sum_{j=1}^n\Theta_j^2
\end{aligned}\end{equation}
其中,$cost_1(\theta^Tx^{(i)})$对应$y=1$; $cost_0(\theta^Tx^{(i)})$对应$y=0$


\subsection{Gaussian Kernel}
\begin{equation}\begin{aligned}
	f_i = similarity(x, l^{(i)}) = exp(-\frac{\sum_{j=1}^n(x_j - l_j^{(i)})^2}{2\sigma^2})
\end{aligned}\end{equation}
\begin{enumerate}
	\item 当$x \approx l^{(i)}$时,$f_i=exp(-\frac{\approx 0^2}{2\sigma^2}) \approx 1$
	\item 当$x$远离$l^{(i)}$时,$f_i=exp(-\frac{inf^2}{2\sigma^2}) \approx 1$
	\item 3
\end{enumerate}





\subsection{SVM中,$C$与$\sigma^2$对欠拟合或过拟合的影响}
\begin{enumerate}
	\item C:C过大:低偏差,高方差;C过小:高偏差,第方差
	\item $\lambda^2$: 过大: 高偏差,低方差;过小: 低偏差,高方差
\end{enumerate}



\subsection{如何选项使用Logistic Regression还是 SVM}
\begin{enumerate}
	\item $n$很大时,使用Logistic Regression或无kernel(即linear kernel)的SVM
	\item $n$很小,$m$中等:使用Gaussian kernel的SVM
	\item $n$很小,$m$很大:增加特征,并使使用Logistic Regression或无kernel(即linear kernel)的SVM
	\item 神经网络可能更加适合,但是使用神经网络训练需要耗费的时间较长
\end{enumerate}
