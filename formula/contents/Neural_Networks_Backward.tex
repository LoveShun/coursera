\subsection{神经网络--后向算法}

\subsubsection{输出层结果:$a^L$}
\begin{equation}
	a^L = 
		\left(\begin{matrix}
			a_1^{(L)(1)} & a_2^{(L)(1)} & a_3^{(L)(1)} & \dots & a_{s_L}^{(L)(1)} \\
			a_1^{(L)(2)} & a_2^{(L)(2)} & a_3^{(L)(2)} & \dots & a_{s_L}^{(L)(2)} \\
			a_1^{(L)(3)} & a_2^{(L)(3)} & a_3^{(L)(3)} & \dots & a_{s_L}^{(L)(3)} \\
			\vdots & \vdots & \vdots & \ddots & \vdots \\
			a_1^{(L)(m)} & a_2^{(L)(m)} & a_3^{(L)(m)} & \dots & a_{s_L}^{(L)(m)} \\
		\end{matrix}\right)_{m,s_L}
\end{equation}

\subsubsection{格式化后的Y}
\begin{equation}\begin{aligned}
	Y &= \left(\begin{matrix}
	        0 & 0 & 0 & \dots & 0 & 1 \\
	        0 & 1 & 0 & \dots & 0 & 0 \\
	        0 & 0 & 1 & \dots & 0 & 0 \\
	        \vdots & \vdots & \vdots & \vdots & \ddots & \vdots \\
	        1 & 0 & 0 & \dots & 0 & 0 \\
		\end{matrix}\right)_{m,s_L}
\end{aligned}\end{equation}

\subsubsection{$\delta^{L}$}
\begin{equation}\begin{aligned}
	\delta^{L} &= a^{L} - y \\
		&= \left(\begin{matrix}
			a_1^{(L)} \\ a_2^{(L)} \\ a_3^{(L)} \\ \vdots \\ a_{s_L}^{(L)}
		\end{matrix}\right) - 
		\left(\begin{matrix}
	        1 \\ 0 \\ 0 \\ \vdots \\ 0 \\ 0 
		\end{matrix}\right)
\end{aligned}\end{equation}
\begin{itemize}
	\item 此时,$\delta^{L}, a^{L}, y$表示的是均向量(不是矩阵)
\end{itemize}

% \begin{equation}\begin{aligned}
% 	\Delta^{L} &= a^{L} - Y \\
% 	&= 	
% 		\left(\begin{matrix}
% 			a_1^{(L)(1)} & a_2^{(L)(1)} & a_3^{(L)(1)} & \dots & a_{s_L}^{(L)(1)} \\
% 			a_1^{(L)(2)} & a_2^{(L)(2)} & a_3^{(L)(2)} & \dots & a_{s_L}^{(L)(2)} \\
% 			a_1^{(L)(3)} & a_2^{(L)(3)} & a_3^{(L)(3)} & \dots & a_{s_L}^{(L)(3)} \\
% 			\vdots & \vdots & \vdots & \ddots & \vdots \\
% 			a_1^{(L)(m)} & a_2^{(L)(m)} & a_3^{(L)(m)} & \dots & a_{s_L}^{(L)(m)} \\
% 		\end{matrix}\right) - 
% 		 \left(\begin{matrix}
% 	        0 & 0 & 0 & \dots & 0 & 1 \\
% 	        0 & 1 & 0 & \dots & 0 & 0 \\
% 	        0 & 0 & 1 & \dots & 0 & 0 \\
% 	        \vdots & \vdots & \vdots & \vdots & \ddots & \vdots \\
% 	        1 & 0 & 0 & \dots & 0 & 0 \\
% 		\end{matrix}\right)_{m,s_L}
% \end{aligned}\end{equation}
% \begin{itemize}
% 	\item 此时,$\Delta^{L}, a^{L}, Y$表示的是均是矩阵
% 	\item {\color{red}{从教程看,似乎不用这样算,都用迭代的方式,这大概也是$\delta^{l}$ 与 $\Delta^{l}$表示的意义不一样的来源}}
% \end{itemize}

\subsubsection{$\delta^{L-1}$}
\begin{equation}\begin{aligned}
	\delta^{L-1} &= (\Theta^{(L-1)})^T \delta^{L} .* g^{'}(z^{L-1}) \\
	&= (\Theta^{(L-1)})^T \delta^{L} .* g(z^{L-1}) .* (1-g(z^{L-1}))
\end{aligned}\end{equation}
\begin{enumerate}
	\item 其中,式$g^{'}(z) = g(z)(1-g(z))$,此为sigmoid()函数的特性
	\item 此时,$z^{L-1}$表示的是一个向量(不是矩阵)
	\item 此时不需要舍弃$\delta_0^{L}$,因为根本就没有
\end{enumerate}

% \begin{equation}\begin{aligned}
% 	\Delta^{L-1} &= (\Theta^{(L-1)})^T \Delta^{L} .* g^{'}(z^{L-1}) \\
% 	&= (\Theta^{(L-1)})^T \Delta^{L} .* g(z^{L-1}) .* (1-g(z^{L-1}))
% \end{aligned}\end{equation}
% \begin{enumerate}
% 	\item 此时,$z^{L-1}$表示的是矩阵
% 	\item 与上述的$\delta^{L-1}$相比,此式批量计算的训练集中的所有数据
% 	\item {\color{red}{从教程看,似乎不用这样算,都用迭代的方式,这大概也是$\delta^{l}$ 与 $\Delta^{l}$表示的意义不一样的来源}}
% \end{enumerate}


\subsubsection{$\delta^{l}$($2<=l<=L-2$)}
\begin{equation}\begin{aligned}
	\delta^{l} &= (\Theta^{(l)})^T \delta^{l+1}(2:end) .* g^{'}(z^{l}) \\
	&= (\Theta^{(l)})^T \delta^{l+1}(2:end) .* g(z^{l}) .* (1-g(z^{l}))
\end{aligned}\end{equation}
\begin{enumerate}
	\item 因$a^{(1)}$直接从X得到,不会有误差,故无$\delta^{(1)}$
	\item (2:end)表示舍弃第一个数据$\delta_0^{s_{L-1}}$(Matlab索引从1开始)
	\item 对比于从$a^{l}$到$a^{l+1}$要添加一个$a_0^{l}=1$;从$\delta^{l+1}$到$\delta^{l}$要舍弃一个$\delta_0^{l+1}$
	\item 同样地,此时$z^{l}$表示的是一个向量(不是矩阵)
\end{enumerate}

\subsubsection{$\Delta^{l}$(用迭代的方式计算)}
\begin{enumerate}
	\item 数值计算方式
	\begin{equation}\begin{aligned}
		\Delta_{ij}^{(l)} := \Delta_{ij}^{(l)} + a_j^{(l)} \delta_i^{(l+1)}
	\end{aligned}\end{equation}
	\item 矩阵计算方式
	\begin{equation}\begin{aligned}
		\Delta^{(l)} := \Delta^{(l)} + \delta^{(l+1)} (a^{(l)})^T
	\end{aligned}\end{equation}
\end{enumerate}

\subsubsection{$D_{ij}^{(l)}$}
\begin{enumerate}
	\item $j=0$时
	\begin{equation}
		D_{ij}^{(l)} := \frac{1}{m}\Delta_{ij}^{(l)}
	\end{equation}
	\item $j \neq 0$时
		\begin{equation}
		D_{ij}^{(l)} := \frac{1}{m}(\Delta_{ij}^{(l)} + \Theta_{ij}^{(l)})
	\end{equation}
\end{enumerate}

\subsubsection{$\frac{\partial{J(\Theta)}}{\partial{\Theta_{ij}^{(l)}}}$}
\begin{equation}
	\frac{\partial{J(\Theta)}}{\partial{\Theta_{ij}^{(l)}}} = D_{ij}^{(l)}
\end{equation}

\subsubsection{$\delta^{l}$ 与 $\Delta^{l}$的区别与联系}













