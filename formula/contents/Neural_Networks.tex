\subsection{神经网络}
	

\subsection{前向算法}
\subsubsection{各矩阵形状}

最初始的数据X为:
\begin{equation} \begin{aligned}
	X & = \left(\begin{matrix}
			(x^{(1)})^T \\ (x^{(2)})^T \\ (x^{(3)})^T \\ \vdots \\ (x^{(m)})^T \\
		\end{matrix}\right) \\
	& = \left( \begin{matrix}
			x_1^{(1)} & x_2^{(1)} & x_3^{(1)} & \dots & x_n^{(1)} \\
			x_1^{(2)} & x_2^{(2)} & x_3^{(2)} & \dots & x_n^{(2)} \\
			x_1^{(3)} & x_2^{(3)} & x_3^{(3)} & \dots & x_n^{(3)} \\
			\vdots    & \vdots    & \vdots    & \ddots & \vdots   \\
			x_1^{(m)} & x_2^{(m)} & x_3^{(m)} & \dots & x_n^{(m)} \\
			\end{matrix}\right) \\
	& = \left(\begin{matrix}
			x_1^{(1)} & x_2^{(1)} & x_3^{(1)} & \dots & x_n^{(1)} \\
			x_1^{(2)} & x_2^{(2)} & x_3^{(2)} & \dots & x_n^{(2)} \\
			x_1^{(3)} & x_2^{(3)} & x_3^{(3)} & \dots & x_n^{(3)} \\
			\vdots    & \vdots    & \vdots    & \ddots & \vdots   \\
			x_1^{(m)} & x_2^{(m)} & x_3^{(m)} & \dots & x_n^{(m)} \\
		\end{matrix}\right)_{m,n}
\end{aligned} \end{equation}

给X测试集的每个数据均添加上$x_0 = 1$后变成$a^{(1)}$:
\footnote{从测试集X得到$a^{(1)}$不需要经过sigmoid()函数,只要在X中添加$x_0 = 1$即可}
\begin{equation}\begin{aligned}
	a^{(1)} &= \left(\begin{matrix} (x_0)_{m*1}, X\end{matrix}\right) = \left(\begin{matrix} 1_{m*1}, X\end{matrix}\right)
	\\ & = \left(\begin{matrix}
		1 & x_1^{(1)} & x_2^{(1)} & x_3^{(1)} & \dots & x_n^{(1)} \\
		1 & x_1^{(2)} & x_2^{(2)} & x_3^{(2)} & \dots & x_n^{(2)} \\
		1 & x_1^{(3)} & x_2^{(3)} & x_3^{(3)} & \dots & x_n^{(3)} \\
		\vdots        & \vdots    & \vdots    & \vdots    & \ddots & \vdots   \\
		1 & x_1^{(m)} & x_2^{(m)} & x_3^{(m)} & \dots & x_n^{(m)} \\
	\end{matrix}\right)_{m,(n+1)}
\end{aligned}\end{equation}

\begin{equation}
\Theta^{(1)} = 
	\left(\begin{matrix}
		\theta_{10}^{(1)} & \theta_{11}^{(1)} & \theta_{12}^{(1)} & \dots & \theta_{1,s_1}^{(1)} \\
		\theta_{20}^{(1)} & \theta_{21}^{(1)} & \theta_{22}^{(1)} & \dots & \theta_{2,s_1}^{(1)} \\
		\theta_{30}^{(1)} & \theta_{31}^{(1)} & \theta_{32}^{(1)} & \dots & \theta_{3,s_1}^{(1)} \\
		\vdots    & \vdots    & \vdots    & \ddots & \vdots   \\
		\theta_{s_{2}0}^{(j)} & \theta_{s_{2}1}^{(j)} & \theta_{s_{2}2}^{(j)} & \dots & \theta_{s_{2},s_{1}}^{(1)}
	\end{matrix}\right)_{(s_{2},s_1+1)=(s_{2},n+1)}
\end{equation}

$a^{(1)}$与$\Theta^{(1)}$计算后等到得到$z^{(2)}=\Theta^{(1)}(a^{(1)})^T$
\footnote{从$a^{(1)}$得到$a^{(2)}$需要经过sigmoid()函数,后续的从$a^{(j)}$得到$a^{(j+1)}$均需要经过sigmoid()函数}
\begin{equation}\begin{aligned}
	z^{(2)} &= (((a^{(1)}) \Theta^{(1)})_{s_{2},n+1}^T)_{(n+1, m)}
		\\ &= 
		  \left(\begin{matrix}
				1 & x_1^{(1)} & x_2^{(1)} & x_3^{(1)} & \dots & x_n^{(1)} \\
				1 & x_1^{(2)} & x_2^{(2)} & x_3^{(2)} & \dots & x_n^{(2)} \\
				1 & x_1^{(3)} & x_2^{(3)} & x_3^{(3)} & \dots & x_n^{(3)} \\
				\vdots        & \vdots    & \vdots    & \vdots    & \ddots & \vdots   \\
				1 & x_1^{(m)} & x_2^{(m)} & x_3^{(m)} & \dots & x_n^{(m)} \\
			\end{matrix}\right)
			\left(\begin{matrix}
				\theta_{10}^{(1)} & \theta_{11}^{(1)} & \theta_{12}^{(1)} & \dots & \theta_{1,n}^{(1)} \\
				\theta_{20}^{(1)} & \theta_{21}^{(1)} & \theta_{22}^{(1)} & \dots & \theta_{2,n}^{(1)} \\
				\theta_{30}^{(1)} & \theta_{31}^{(1)} & \theta_{32}^{(1)} & \dots & \theta_{3,n}^{(1)} \\
				\vdots    & \vdots    & \vdots    & \ddots & \vdots   \\
				\theta_{s_{2},0}^{(j)} & \theta_{s_{2},1}^{(j)} & \theta_{s_{2},2}^{(j)} & \dots & \theta_{s_{2},n}^{(1)}
			\end{matrix}\right)^T
		\\ &= 
		  \left(\begin{matrix}
				1 & x_1^{(1)} & x_2^{(1)} & x_3^{(1)} & \dots & x_n^{(1)} \\
				1 & x_1^{(2)} & x_2^{(2)} & x_3^{(2)} & \dots & x_n^{(2)} \\
				1 & x_1^{(3)} & x_2^{(3)} & x_3^{(3)} & \dots & x_n^{(3)} \\
				\vdots        & \vdots    & \vdots    & \vdots    & \ddots & \vdots   \\
				1 & x_1^{(m)} & x_2^{(m)} & x_3^{(m)} & \dots & x_n^{(m)} \\
			\end{matrix}\right)
			\left(\begin{matrix}
				\theta_{10}^{(1)} & \theta_{20}^{(1)} & \theta_{30}^{(1)} & \dots & \theta_{s_{2},0}^{(1)} \\
				\theta_{11}^{(1)} & \theta_{21}^{(1)} & \theta_{31}^{(1)} & \dots & \theta_{s_{2},1}^{(1)} \\
				\theta_{12}^{(1)} & \theta_{22}^{(1)} & \theta_{32}^{(1)} & \dots & \theta_{s_{2},2}^{(1)} \\
				\theta_{13}^{(1)} & \theta_{23}^{(1)} & \theta_{33}^{(1)} & \dots & \theta_{s_{2},3}^{(1)} \\
				\vdots    & \vdots    & \vdots    & \ddots & \vdots   \\
				\theta_{1,n}^{(1)} & \theta_{2,n}^{(1)} & \theta_{3,n}^{(1)} & \dots & \theta_{s_{2},n}^{(1)}
			\end{matrix}\right)
		% \\ &=
			% \left(\begin{matrix}
			% 	\theta_{10}^{(1)} & \theta_{11}^{(1)} & \theta_{12}^{(1)} & \dots & \theta_{1,n}^{(1)} \\
			% 	\theta_{20}^{(1)} & \theta_{21}^{(1)} & \theta_{22}^{(1)} & \dots & \theta_{2,n}^{(1)} \\
			% 	\theta_{30}^{(1)} & \theta_{31}^{(1)} & \theta_{32}^{(1)} & \dots & \theta_{3,n}^{(1)} \\
			% 	\vdots    & \vdots    & \vdots    & \ddots & \vdots   \\
			% 	\theta_{s_{2}0}^{(j)} & \theta_{s_{2}1}^{(j)} & \theta_{s_{2}2}^{(j)} & \dots & \theta_{s_{2},n}^{(1)}
			% \end{matrix}\right)  \left(\begin{matrix}
			% 	1 & 1 & 1 & \dots & 1 \\
			% 	x_1^{(1)} & x_1^{(2)} & x_1^{(3)} & \dots & x_1^{(m)} \\
			% 	x_2^{(1)} & x_2^{(2)} & x_2^{(3)} & \dots & x_2^{(m)} \\
			% 	x_3^{(1)} & x_3^{(2)} & x_3^{(3)} & \dots & x_3^{(m)} \\
			% 	\vdots & \vdots & \vdots & \ddots & \vdots \\
			% 	x_n^{(1)} & x_n^{(2)} & x_n^{(3)} & \dots & x_n^{(m)} \\
			% \end{matrix}\right)
		\\ &=
			\left(\begin{matrix}
				z^{(2)(1)} \\ z^{(2)(2)} \\ z^{(2)(3)} \\ \vdots \\ z^{(2)(m)} 
			\end{matrix}\right)
		\\ &= 
			\left(\begin{matrix}
				z_1^{(2)(1)} & z_2^{(2)(1)} & z_3^{(2)(1)} & \dots & z_{s_2}^{(2)(1)} \\
				z_1^{(2)(2)} & z_2^{(2)(2)} & z_3^{(2)(2)} & \dots & z_{s_2}^{(2)(2)} \\
				z_1^{(2)(3)} & z_2^{(2)(3)} & z_3^{(2)(3)} & \dots & z_{s_2}^{(2)(3)} \\
				\vdots & \vdots & \vdots & \ddots & \vdots \\
				z_1^{(2)(m)} & z_2^{(2)(m)} & z_3^{(2)(m)} & \dots & z_{s_2}^{(2)(m)} \\
			\end{matrix}\right)
\end{aligned}\end{equation}

$s_1 = n$,因为$s_k$不算bias unit。\\
$size(a^{(j)}) = (s_j+1)*1$,\\
$size(a^{(j+1)}) = (s_{j+1}+1)*1$\\ 
$a^{(j+1)} = g(X\Theta)$,\\
故$a^{(j)}$、$\Theta^{(j)}$与$a^{(j+1)}$的矩阵大小关系为:\\
$a^{(j)}$:
$\Theta^{(j)}$:
$a^{(j+1)}$:


% 之前写的,暂放后面
\newpage

\begin{equation}\begin{aligned}
	temp^{(1)} = \left(\begin{matrix} 1 & X \end{matrix}\right)
	= \left( \begin{matrix}
			1 & x_1^{(1)} & x_2^{(1)} & x_3^{(1)} & \dots & x_n^{(1)} \\
			1 & x_1^{(2)} & x_2^{(2)} & x_3^{(2)} & \dots & x_n^{(2)} \\
			1 & x_1^{(3)} & x_2^{(3)} & x_3^{(3)} & \dots & x_n^{(3)} \\
			\vdots & \vdots    & \vdots    & \vdots    & \ddots & \vdots   \\
			1 & x_1^{(m)} & x_2^{(m)} & x_3^{(m)} & \dots & x_n^{(m)}
		\end{matrix}\right)_{m*(n+1)}
\end{aligned}\end{equation}


\begin{equation}
	a^{(1)} = g(z^{(1)})
\end{equation}

\begin{equation}\begin{aligned}
	z^{(j+1)} = \left(\begin{matrix} 1 & a^{(j)} \end{matrix}\right)
	= \left( \begin{matrix}
			1 & a_1^{(1)} & a_2^{(1)} & a_3^{(1)} & \dots & a_{s_j}^{(1)} \\
			1 & a_1^{(2)} & a_2^{(2)} & a_3^{(2)} & \dots & a_{s_j}^{(2)} \\
			1 & a_1^{(3)} & a_2^{(3)} & a_3^{(3)} & \dots & a_{s_j}^{(3)} \\
			\vdots & \vdots    & \vdots    & \vdots    & \ddots & \vdots   \\
			1 & a_1^{(m)} & a_2^{(m)} & a_3^{(m)} & \dots & a_{s_j}^{(m)}
		\end{matrix}\right)_{m*(s_j+1)}
\end{aligned}\end{equation}

\begin{equation}
	a^{(j+1)} = g(z^{(j+1)})
\end{equation}

最初始的y
\begin{equation} \begin{aligned}
	y & = \left(\begin{matrix}
			y^{(1)} \\ y^{(2)} \\ y^{(3)} \\ \vdots \\ y^{(m)} \\
		\end{matrix}\right)_{m*1}
\end{aligned} \end{equation}

为进行矩阵运算,要将其转化为如下形式:\\
PS: 表示为y的值所在的索引位置值为1,其他位置均为0
\begin{equation}\begin{aligned}
	y &= \left(\begin{matrix}
	        0 & 0 & 0 & \dots & 0 & 1 \\
	        0 & 1 & 0 & \dots & 0 & 0 \\
	        0 & 0 & 1 & \dots & 0 & 0 \\
	        \vdots & \vdots & \vdots & \vdots & \ddots & \vdots \\
	        1 & 0 & 0 & \dots & 0 & 0 \\
		\end{matrix}\right)_{m*s_L}
\end{aligned}\end{equation}
上式$m*s_L$中的$s_L$表示共有$s_L$个分类器,$s_L$表示的是输出层的unit数




$\Theta^{(j)}$ 的形状为: $s_{j+1}*(s_j+1)$ \\

\begin{equation}
\Theta^{(j)} = 
	\left(\begin{matrix}
		\theta_{10}^{(j)} & \theta_{11}^{(j)} & \theta_{12}^{(j)} & \dots & \theta_{1,s_j}^{(j)} \\
		\theta_{20}^{(j)} & \theta_{21}^{(j)} & \theta_{22}^{(j)} & \dots & \theta_{2,s_j}^{(j)} \\
		\theta_{30}^{(j)} & \theta_{31}^{(j)} & \theta_{32}^{(j)} & \dots & \theta_{3,s_j}^{(j)} \\
		\vdots    & \vdots    & \vdots    & \ddots & \vdots   \\
		\theta_{s_{j+1,0}}^{(j)} & \theta_{s_{j+1,1}}^{(j)} & \theta_{s_{j+1,2}}^{(j)} & \dots & \theta_{s_{j+1},s_j}^{(j)}
	\end{matrix}\right)_{(s_{j+1},s_j)}
\end{equation}






\subsubsection{数值计算方法}
\begin{enumerate}
\item 单个unit的特殊情况
\begin{equation}
	a_1^{(2)} = g(z_1^{(2)}) = g(a_0^{(1)}\theta_{10}^{(1)}
	+ a_1^{(1)}\theta_{11}^{(1)}
	+ a_2^{(1)}\theta_{12}^{(1)}
	+ \dots
	+ a_{s1}^{(1)}\theta_{1s1}^{(1)})
\end{equation}

\item 单个unit的一般情况
\begin{equation} \begin{aligned}
	a_i^{(j+1)} &= g(z_i^{(j+1)})
	\\ & = g(a_0^{(j)}\theta_{j0}^{(j)} + a_{1}^{(j)}\theta_{j1}^{(j)} + a_2^{(j)}\theta_{j2}^{(j)} + \dots + a_{s1}^{(j)}\theta_{js1}^{(j)})
	\\ &= g(\sum_{k=0}^{s_l}{a_k^{(j)}\theta_{jk}^{(j)}})
\end{aligned}\end{equation}
\end{enumerate}

\subsubsection{矩阵计算方法}
\begin{enumerate}
\item 单个unit的特殊情况
\begin{equation}
	a_1^{(2)} = g(\Theta^{(1)} a^{(1)})
\end{equation}

\item 单个unit的一般情况
\begin{equation}
	a_i^{(j+1)} = g(z_i^{(j+1)}) = g(\Theta^{(j)} a^{(j)})
\end{equation}

% \item 单Layer的普通情况 \\
% {\color{red}{1. i是否从0开始待确定}} \\
% {\color{red}{2. 此公式的正确性待确定,似乎没见过计算整Layer的情况}}
% \begin{equation} \begin{aligned}

% \end{aligned}\end{equation}
\end{enumerate}







