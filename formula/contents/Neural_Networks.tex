\subsection{神经网络}
	

\subsection{前向算法}
\subsubsection{各矩阵形状}
\begin{enumerate}

\item X
\begin{equation} \begin{aligned}
	X & = \left(\begin{matrix}
			x^{(0)} \\ x^{(1)} \\ x^{(2)} \\ x^{(3)} \\ \vdots \\ x^{(m)} \\
		\end{matrix}\right) \\
	& = \left( \begin{matrix}
			x_0^{(1)} & x_1^{(1)} & x_2^{(1)} & x_3^{(1)} & \dots & x_n^{(1)} \\
			x_0^{(2)} & x_1^{(2)} & x_2^{(2)} & x_3^{(2)} & \dots & x_n^{(2)} \\
			x_0^{(3)} & x_1^{(3)} & x_2^{(3)} & x_3^{(3)} & \dots & x_n^{(3)} \\
			\vdots    & \vdots    & \vdots    & \vdots    & \ddots & \vdots   \\
			x_0^{(m)} & x_1^{(m)} & x_2^{(m)} & x_3^{(m)} & \dots & x_n^{(m)} \\
			\end{matrix}\right) \\
	& = \left(\begin{matrix}
			1 & x_1^{(1)} & x_2^{(1)} & x_3^{(1)} & \dots & x_n^{(1)} \\
			1 & x_1^{(2)} & x_2^{(2)} & x_3^{(2)} & \dots & x_n^{(2)} \\
			1 & x_1^{(3)} & x_2^{(3)} & x_3^{(3)} & \dots & x_n^{(3)} \\
			\vdots    & \vdots    & \vdots    & \vdots    & \ddots & \vdots   \\
			1 & x_1^{(m)} & x_2^{(m)} & x_3^{(m)} & \dots & x_n^{(m)} \\
		\end{matrix}\right)_{n*m}
\end{aligned} \end{equation}

\item y
最初始的y
\begin{equation} \begin{aligned}
	X & = \left(\begin{matrix}
			y^{(0)} \\ y^{(1)} \\ y^{(2)} \\ y^{(3)} \\ \vdots \\ y^{(m)} \\
		\end{matrix}\right)
\end{aligned} \end{equation}

为进行矩阵运算,要将其转化为:
\begin{equation}\begin{aligned}
	y &= \left(\begin{matrix}

		\end{matrix}\right)
\end{aligned}\end{equation}


\item $\Theta^{(j)}$  \\

$\Theta^{(j)}$ 的形状为: $s_{j+1}*(s_j+1)$ \\

\begin{equation}
\Theta^{(j)} = 
	\left(\begin{matrix}
		\theta_{10}^{(j)} & \theta_{11}^{(j)} & \theta_{12}^{(j)} & \dots & \theta_{1,s_j}^{(j)} \\
		\theta_{20}^{(j)} & \theta_{21}^{(j)} & \theta_{22}^{(j)} & \dots & \theta_{2,s_j}^{(j)} \\
		\theta_{30}^{(j)} & \theta_{31}^{(j)} & \theta_{32}^{(j)} & \dots & \theta_{3,s_j}^{(j)} \\
		\vdots    & \vdots    & \vdots    & \ddots & \vdots   \\
		\theta_{s_{j+1,0}}^{(j)} & \theta_{s_{j+1,1}}^{(j)} & \theta_{s_{j+1,2}}^{(j)} & \dots & \theta_{s_{j+1},s_j}^{(j)}
	\end{matrix}\right)_{(s_{j+1},s_j+1)}
\end{equation}




\end{enumerate}



\subsubsection{数值计算方法}
\begin{enumerate}
\item 单个unit的特殊情况
\begin{equation}
	a_1^{(2)} = g(z_1^{(2)}) = g(a_0^{(1)}\theta_{10}^{(1)}
	+ a_1^{(1)}\theta_{11}^{(1)}
	+ a_2^{(1)}\theta_{12}^{(1)}
	+ \dots
	+ a_{s1}^{(1)}\theta_{1s1}^{(1)})
\end{equation}

\item 单个unit的一般情况
\begin{equation} \begin{aligned}
	a_i^{(j)} &= g(z_i^{(j)})
	\\ & = g(a_0^{(j-1)}\theta_{(j-1)0}^{(j-1)} + a_{1}^{(j-1)}\theta_{(j-1)1}^{(j-1)} + a_2^{(j-1)}\theta_{(j-1)2}^{(j-1)} + \dots + a_{s1}^{(j-1)}\theta_{(j-1)s1}^{(j-1)})
	\\ &= g(\sum_{k=0}^{s_l}{a_k^{(j-1)}\theta_{(j-1)k}^{(j-1)}})
\end{aligned}\end{equation}

\item 单Layer的普通情况 \\
{\color{red}{1. i是否从0开始待确定}} \\
{\color{red}{2. 此公式的正确性待确定,似乎没见过计算整Layer的情况}}
\begin{equation} \begin{aligned}
	a^{(j)} &= \sum_{i=0}^{s_{(j)}}{a_i^{(j-1)}}
		\\ &= \sum_{i=0}^{s_{(j)}}g(\sum_{k=0}^{s_j}{a_k^{(j-1)}\theta_{(j-1)k}^{(j-1)}}))
\end{aligned}\end{equation}
\end{enumerate}

\subsubsection{矩阵计算方法}
\begin{enumerate}
\item 单个unit的特殊情况
\begin{equation}
	a_1^{(2)} = g(\Theta^{(1)} a^{(1)})
\end{equation}

\item 单个unit的一般情况
\begin{equation}
	a_i^{(j)} = g(z_i^{(j)}) = g(\Theta^{(j-1)} a^{(j-1)})
\end{equation}

% \item 单Layer的普通情况 \\
% {\color{red}{1. i是否从0开始待确定}} \\
% {\color{red}{2. 此公式的正确性待确定,似乎没见过计算整Layer的情况}}
% \begin{equation} \begin{aligned}

% \end{aligned}\end{equation}
\end{enumerate}







